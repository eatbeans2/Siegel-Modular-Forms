\documentclass[11pt, oneside]{article}   	% use "amsart" instead of "article" for AMSLaTeX format
\usepackage{geometry}                		% See geometry.pdf to learn the layout options. There are lots.
\geometry{letterpaper}                   		% ... or a4paper or a5paper or ... 
%\geometry{landscape}                		% Activate for rotated page geometry
%\usepackage[parfill]{parskip}    		% Activate to begin paragraphs with an empty line rather than an indent
\usepackage{graphicx}				% Use pdf, png, jpg, or eps§ with pdflatex; use eps in DVI mode

\usepackage{mathtools}								% TeX will automatically convert eps --> pdf in pdflatex		
\usepackage{amssymb}

%SetFonts

%SetFonts


\title{Distinguished Representatives of Binary Quadratic Forms Under Congruence Subgroups}
\author{Beau Horenberger}
%\date{}							% Activate to display a given date or no date

\begin{document}
\maketitle
\section*{Diamond and Shurman 1.2.3: Proof that  $[ SL_2(\mathbb{Z} ):\Gamma_0 (N) ] = N\prod_{p\vert N}(1+1/p)$.}

\subsection*{a) Let $p$ be a prime and let $e$ be a positive integer. Show that $\vert SL_2(\mathbb{Z}/p^e \mathbb{Z})\vert = p^{3e}(1-1/p^2)$.}

\textbf{Proof:}

The proof is through induction.

Base case: Let $e=1$. Then the map $GL_2(\mathbb{Z}/p\mathbb{Z})\rightarrow (\mathbb{Z}/p\mathbb{Z})^*$ via the determinant map has kernel $SL_2(\mathbb{Z}/p\mathbb{Z})$. By the first isomorphism theorem $GL_2(\mathbb{Z}/p\mathbb{Z})/SL_2(\mathbb{Z}/p\mathbb{Z})$ is isomorphic to $(\mathbb{Z}/p\mathbb{Z})^*$, and it can be calculated that $\vert GL_2(\mathbb{Z}/p\mathbb{Z}) \vert = (p^2-1)(p^2-p)$. Then it follows $\vert SL_2(\mathbb{Z}/p^e \mathbb{Z})\vert = p^{3}(1-1/p^2)$ in order for the isomorphism to be true.

Inductive step: Now we consider $SL_2(\mathbb{Z}/p^{e+1}\mathbb{Z})\rightarrow SL_2(\mathbb{Z}/p^e\mathbb{Z})$, where $A\rightarrow A\bmod{p^e}$. This map is surjective, and the kernel has size ???.

\subsection*{b) Show that $\vert SL_2(\mathbb{Z}/N \mathbb{Z})\vert=N^3\prod_{p\vert N} (1-1/p^2)$, so this is the index $[SL_2(\mathbb{Z}):\Gamma (N)]$.}

\textbf{Proof:}

This follows from the above and the Chinese Remainder Theorem.

\subsection*{c) Show that the map $\Gamma_1 (N)\rightarrow (\mathbb{Z}/N\mathbb{Z})$ given by $\bigl( \begin{smallmatrix}a & b\\ c & d\end{smallmatrix}\bigr) \rightarrow b \bmod{N}$ surjects and has kernel $\Gamma (N)$.}

\textbf{Proof:}

For surjection, note that for all $x \in (\mathbb{Z}/N\mathbb{Z})$, we have  $\bigl( \begin{smallmatrix}1 & x\\ 0 & 1\end{smallmatrix}\bigr)\rightarrow x$. Thus, we have surjectivity.

For the kernel, we suppose $\bigl( \begin{smallmatrix}1 & x\\ 0 & 1\end{smallmatrix}\bigr)\rightarrow 0$. Then $x\equiv 0 \bmod N$, so $\bigl( \begin{smallmatrix}1 & x\\ 0 & 1\end{smallmatrix}\bigr) \equiv \bigl( \begin{smallmatrix}1 & 0\\ 0 & 1\end{smallmatrix}\bigr) \bmod N$. Thus our kernel is $\Gamma (N)$. $\square$

Note this also implies $\Gamma (N)$ is normal in $\Gamma_1 (N)$, and that $\Gamma_1 (N)/\Gamma (N)$ is isomorphic to $(\mathbb{Z}/N\mathbb{Z})$. Thus, $[\Gamma_1 (N):\Gamma (N)]=N$

\subsection*{d) Show that the map $\Gamma_0 (N)\rightarrow (\mathbb{Z}/N\mathbb{Z})^*$ given by $\bigl( \begin{smallmatrix}a & b\\ c & d\end{smallmatrix}\bigr) \rightarrow d \bmod{N}$ surjects and has kernel $\Gamma_1 (N)$.}

\textbf{Proof:}

Let $d\in (\mathbb{Z}/N\mathbb{Z})^*$ be given. Then $\bigl( \begin{smallmatrix}0 & 0\\ 0 & d\end{smallmatrix}\bigr)\rightarrow d$. So we have surjection.

For the kernel, let $\bigl( \begin{smallmatrix}a & b\\ c & d\end{smallmatrix}\bigr)\rightarrow 1$. Then $d\equiv 1 \bmod N$, so $\bigl( \begin{smallmatrix}a & b\\ c & d\end{smallmatrix}\bigr)\equiv \bigl( \begin{smallmatrix}a & b\\ 0 & 1\end{smallmatrix}\bigr)\bmod N$. But $ad\equiv 1\bmod N$, so $a\equiv 1 \bmod N$. So $\bigl( \begin{smallmatrix}a & b\\ c & d\end{smallmatrix}\bigr) \equiv \bigl( \begin{smallmatrix}1 & b\\ 0 & 1\end{smallmatrix}\bigr)\bmod N$. Then we have our kernel as $\Gamma_1 (N)$. $\square$

By the logic in c), this also implies $[\Gamma_0 (N):\Gamma_1 (N)]=\phi(N)$

\subsection*{e) Show that $[SL_2(\mathbb{Z}):\Gamma_0 (N)]=N\prod_{p\vert N} (1+1/p)$}

\section*{Coset representatives for congruence subgroups computed algebraically}

\subsection*{Claim: There is a bijection between the cosets of $\Gamma/\Gamma_0 (N)$ and the projective space $\mathbb{P}^1(\mathbb{Z}/N\mathbb{Z})$}

First note that $\mathbb{P}^1(\mathbb{Z}/N\mathbb{Z})$ is defined as the equivalence classes of $P(\mathbb{Z}/N\mathbb{Z})=\{(a,b)\in (\mathbb{Z}/N\mathbb{Z})^2\vert \exists c,d \in (\mathbb{Z}/N\mathbb{Z}) \text{ so } ad-bc=1\}$ under the action of units, $\lambda (a,b)=(\lambda a, \lambda b)$. Such equivalence classes are denoted $(a:b)$

\textbf{Proof:}

Consider the map $SL_2(\mathbb{Z})/\Gamma_0 (N)\rightarrow \mathbb{P}^1(\mathbb{Z}/N\mathbb{Z})$ given by $\Gamma_0 (N)\bigl( \begin{smallmatrix}a & b\\ c & d\end{smallmatrix}\bigr)\rightarrow (c:d)$. Observe this is well-defined; if $\bigl( \begin{smallmatrix}A & B\\ C & D\end{smallmatrix}\bigr)\in \Gamma_0 (N)$ and $\bigl( \begin{smallmatrix}a & b\\ c & d\end{smallmatrix}\bigr)$ is a member of a coset, then $\bigl( \begin{smallmatrix}A & B\\ C & D\end{smallmatrix}\bigr)\bigl( \begin{smallmatrix}a & b\\ c & d\end{smallmatrix}\bigr)$ has values $Ca+Dc$ and $Cb+Dd$ in its bottom row, corresponding to $(cD:dD)$, since $C\equiv 0 \bmod{N}$. And since the determinant implies $D$ is a unit, we see that every member of a coset maps to the same equivalence class. Thus, we are well-defined.

For surjection, simply give a $(c:d)$, and a corresponding $\bigl( \begin{smallmatrix}a & b\\ c & d\end{smallmatrix}\bigr)$ can be solved using the Euclidean algorithm.

For injection, we want to show that if two matrices map to the same equivalence class, then they belong to the same coset. This means we must show $\bigl( \begin{smallmatrix}A & B\\ C & D\end{smallmatrix}\bigr)\rightarrow (C:D)$ and $\bigl( \begin{smallmatrix}a & b\\ c & d\end{smallmatrix}\bigr)\rightarrow (c:d)=(C:D)$ implies $\bigl( \begin{smallmatrix}A & B\\ C & D\end{smallmatrix}\bigr)\bigl( \begin{smallmatrix}a & b\\ c & d\end{smallmatrix}\bigr)^{-1}\in \Gamma_0 (N)$, or $(Cd-cD)\equiv 0\bmod{N}$. By the equality of equivalence classes, we have $c=\lambda C$ and $d=\lambda D$, so we have $(Cd-cD)\equiv(\lambda CD-\lambda CD)\equiv 0 \bmod{N}$, as desired. Thus, we have our bijection. $\square$

\subsection*{Generating Representatives:}

\textbf{Claim: } All members of the projective line $\mathbb{P}^1(\mathbb{Z}/p^n\mathbb{Z})$, with $p$ a prime and $n$ a natural number, of the form $(1:u)$ where $u=0,1,..,p^{n}-1$ and $pu, 1$ where $u=0,1,p^{n-1}-1$ represent different equivalence classes.

\textbf{Proof:}

Suppose $(1:u)=(1:u')$. then $u\equiv\lambda u' \bmod{N}$ and $1\equiv\lambda 1 \bmod{N}$. Thus $\lambda=1,N$, but this implies $u'=u$. Thus, all members where $u\neq u'$ are distinct.

Similarly, all $(pu, 1)$ are distinct.

Then suppose $(1:u)=(pu':1)$. So $1\equiv \lambda pu' \bmod{N}$ and $1\equiv \lambda u \bmod{N}$ Then $pu'\equiv \lambda^{-1} \equiv u \bmod{N}$, and thus $pu'$ is a unit, which is a contradiction. Thus, $(1:u)\neq(pu':1)$. $\square$

Further, note that for $p^n$ the quantity of these distinct equivalence classes is exactly the number of equivalence classes (verified using our bijection and the first part of this paper).

Thus, we have a method for computing coset representatives of $\Gamma/\Gamma_0 (N)$


\section*{TO DO:}

Finish parts a), e) in section one.

Program: Implement Euclidean algorithm and matrix operations. Decide on overarching format (.h files etc for modularity with future additions)

\section*{Useful sources}

Info on fundamental domain: http://www2.math.ou.edu/~kmartin/mfs/

Calculations via fundamental domain: https://wstein.org/edu/Fall2003/252/lectures/09-19-03/index.html

Even more on calculations of fundamental domains: https://wstein.org/edu/Fall2003/252/lectures/09-19-03/fundomain-alg.pdf

Description of coset representatives for congruence subgroups: https://mathoverflow.net/questions/83858/cosets-representatives-of-congruence-subgroups (SOURCE???)

More on Minkowski's reduction theorem: https://math.stackexchange.com/questions/2623427/problem-on-minkowskis-reduction-theory-of-positive-definite-matrix

HUGELY RELEVANT BOOK: Modular forms: A classical and computational introduction, Kilford


\end{document}  